\documentclass{article}

\usepackage{geometry}
\usepackage{hyperref}

\title {Khidmat: Project Title}

\author{
  Abyan Amir\\ aa02820
  \and
  Mohammad Ateeb Ahmed\\ ma02954
  \and
  Gulraiz Askari\\ ga02197
  \and
  Moonis Rasheed\\ mr02388
  \and
  Syeda Sahar Fatima\\ sf02969
}
\date{}  

\begin{document}
\maketitle

% Use first person plural (we, us) even if you did the Khidmat individually.

% An introduction of the project, no more than 2 sentences. Provide the highest level of detail only. Other details will come later.
% Typically, "This project is to <short description of porject> for/at <client>."
%This project is to build a testing system to be used for the %entrance examination at Habib University.

This Project is to build basic understanding of Microsoft Word, Excel and PowerPoint as well as some programmming ( Python) among the students of TCF, to whom we are teaching.

% About the client.
%Habib University is a first of its kind liberal arts and science programs. Founded through the largest philanthropic grant to higher education in the history of Pakistan, it is a registered non-profit organization with.
The Citizens Foundation (TCF) is a professionally managed, non-profit organization set up in 1995 by a group of citizens who wanted to bring about positive social change through education. 22 years later, TCF is now one of Pakistan’s leading organizations in the field of education for the less privileged. For further details please visit     \href{https://www.tcf.org.pk/#about
}{here}.
\\
% About the project.
%Striving to admit the most deserving students, Habib University uses \href{https://accuplacer.collegeboard.org}{ACCUPLACER} for its entrance test but wants to move to its own test. Faculty members will contribute questions to a pool and the new examination system will choose and present questions at random from the pool to each test taker. This will ensure that each test taker gets a unique test. For our Khidmat, we will build Habib University's new examination system.

Much of what the students, in the future and as undergraduates, will be focused towards working using the utilities by Microsoft. Most particularly, Microsoft Word, Microsoft PowerPoint, Microsoft Excel, Microsoft Outlook and numerous others but not excluding the online utilities. Not to mention, the opportunities provided by these utilities alone in the market are quite massive and most certainly, rewarding in different manners especially for growth and experience.\\
Furthermore, taking inspiration from the Hour of Code, we realize the importance of students having an awareness of programming, in general, and thereby, possessing a basic skillset. In that regards, the programme has a week-long session, in the end of this whole program, in which students will be equipped with the ability to program on a basic level.  \\


% About the plan of work.
%We will work full time on Habib University's premises under the supervision of their Admission Manager. The goal is to develop, test, and deploy the system by the end of our Khidmat.

\\We have divided the whole student population of TCF Bhittaiabad campus into sections, namely ExMatric ( Those who have done matriculation and are applying for colleges ) , Matric A and Matric B. Each pair of volunteers will be given a section from the above pool, and they will invest 2.5 Hours daily ( from monday to Saturday ), that means a certain pair of volunteers needs to be on campus for a minimum of 2.5 Hours, but that is not the case with some, because they will also be taking Math or English classes for TCF ADP program, and for that those volunteers will be granted certificates.

% Copy-paste this section with necessary modifications for each week.
\newpage % Start the report for each week on a new page.
\section*{Week 1: 26 June -- 1 July, 2018}

% A summary, maximum 2 sentences, of this week's activities.
This week's classes covered: Microsoft Word.\\


\begin{tabular}{|l|l|l|l|}
  \hline
  Item 	& Activity & Time & ID \\\hline\hline
  1	& Held Class & 9 hrs & ma02954 \\\hline
  2	& Held Q\&A session & 6 hrs & ma02954 \\\hline
  3	& Held Class & 9 hrs & ga02197 \\\hline
  4	& Held Q\&A session & 6 hrs & ga02197 \\\hline
  5	& Held Class & 9 hrs & mr02388  \\\hline
  6	& Held Q\&A session & 6 hrs & mr02388  \\\hline
  7	& Held Class & 9 hrs & sf02969  \\\hline
  8	& Held Q\&A session & 6 hrs & sf02969 \\\hline
  9	& Held Class & 6 hrs & aa02820 \\\hline
  10	& Held Q\&A session & 4 hrs & aa02820 \\\hline
\end{tabular}

The total time spent on the Khidmat this week is as follows.

\begin{tabular}{|l|l|}
  \hline
  ID & Total Hours\\\hline\hline
  ma02954 & 15 hours\\\hline
  ga02197 & 15 hours\\\hline
  mr02388  & 15 hours\\\hline
  aa02820  & 10 hours\\\hline
  sf02969 & 15 hours\\\hline
  
\end{tabular}

% Other weeks ...

\newpage
\section*{Conclusion}

% Remind the reader about the project. Summarise your activities over the course of the project.
%Our project was to build a new testing system for Habib University to replace ACCUPLACER for its entrance examination. We started by meeting all the stakeholders to understand their expectations from the new system. We then identified the necessary tools to build the required system and trained ourselves on them. Development and testing were carried out in collaboration with the IT team so that any shortcomings were identified and catered to as we went along. The system was then deployed and officers from the Admissions Team were trained to use it.
Our project was to provide a intermediate level of understanding of Microsoft Office to under-privileged students of TCF Bhittaiabad Campus. These skills will provide them better job opportunities (data entry) in the future and will also be beneficial for them if they decided to pursue higher education.

\newpage
% Show your external supervisor your report, especially the weekly upates; have them sign a printed copy of this page; scan the signed page; and include the scanned page in this document as an image.

I hereby certify that I supervised this Khidmat and that I have read and agree with the weekly updates included in the Khidmat report.\\[50pt]

\noindent\begin{tabular}{@{}p{.6\textwidth}@{\hspace{.1\textwidth}}p{.3\textwidth}}
  \hrulefill &   \hrulefill \\
  Name and signature & Date and place
\end{tabular}

\end{document}
